\section{Was ist Open Data?}

\subsection{Welche Daten sind gemeint?}

\begin{frame}[t]{Welche Daten sind gemeint?}
\begin{itemize}
 \item Staatlich
 \item nicht privat
\end{itemize}
\end{frame}

\subsection{Offene Daten}
\begin{frame}[t]{Was heißt offen?}
 \begin{itemize}
   \item jeder darf nutzen und weiterverbreiten
   \item offenes format, maschinenlesbar
   \item ohne zugangsbeschränkung (registrierung, kosten)
   \item Bei staatlichen: proaktiv \& zeitnah
 \end{itemize}
\end{frame}

\subsection{Warum Open Data?}
\begin{frame}[t]{Warum Open Data?}
 \begin{itemize}
  \item politische entscheidungen nachvollziehen $\rightarrow$ fahrrad-einbahnschilder
  \item komplexe sachverhalte besser verstehen $\rightarrow$ ???
  \item innovationen fördern $\rightarrow$ online-karten für sehbehinderte
 \end{itemize}
\end{frame}

\begin{frame}{Warum Open Data?}
 "Ultimately, a more informed citizen is a more empowered citizen. In a modern democracy citizens rightly expect government to show where the money has been spent and what the results have been"
 
 - Chief Secretary to the Treasury, 2009, p.25 (UK)
\end{frame}
\section{Weiterführendes}
  \begin{thebibliography}{10}
  \begin{frame}[t]{Weiterführendes}   
   \bibitem{flags}
    Flaggen aus wikipedia
     \newblock {}
     \newblock \url{http://de.wikipedia.org/wiki/Datei:Flag_of_Germany.svg}, \url{http://de.wikipedia.org/wiki/Datei:Flag_of_North_Rhine-Westphalia_(state).svg}, \url{http://de.wikipedia.org/wiki/Datei:DEU_Paderborn_COA.svg}
   \bibitem{Laefi}
    OK Lab Hamburg
    \newblock {\em Länderfinanzausgleich visualisieren}
    \newblock \url{http://public.tableausoftware.com/profile/\#!/vizhome/LFA_v2/Dashboard1} 
   \bibitem{Heilbronn}
    OK Lab Heilbronn
    \newblock {\em Was steckt in meinem Leitungswasser?}
    \newblock \url{http://opendatalab.de/projects/trinkwasser/}
   \bibitem{Wheelmap}
    Wheelmap
    \newblock{eine Online-Karte [...] rollstuhlgerechter Orte}
    \newblock \url{http://opendatalab.de/projects/trinkwasser/}
   
\end{frame}


\begin{frame}[t]{Weiterführendes}   
   \bibitem{Berlin}
    OK Lab Berlin
    \newblock {Bürger baut Stadt}
    \newblock \url{http://buergerbautstadt.de/}  
   \bibitem{Dresden}
    OK Lab Dresden
    \newblock {\em Freie Parkplätze Dresden}
    \newblock \url{http://ubahn.draco.uberspace.de/opendata/ui/}
   \bibitem{codefor.de}
    CODE for Germany
    \newblock {Stadt<entwickler /> nutzen offene Daten, um ihre Stadt zu verbessern.}
    \newblock \url{http://codefor.de/}
   \bibitem{Kloiber2013}
    Julia Kloiber
    \newblock {\em Open Data -- und was hat das mit mir zu tun? \href{http://www.slideshare.net/juliakloiber/open-data-und-was-hat-das-mit-mir-zu-tun-republica-2013}{(Link zu slideshare)}}.
    \newblock Re:publica 2013.
   

   
\end{frame}

  \end{thebibliography}